\section{Comparaciones}

\subsection{Exacto - Goloso}

Como vimos anteriormente el algoritmo Exacto tiene complejidad exponencial mientras que el Goloso polinomial. Es decir que para valores de entrada grandes el Goloso va a costar mucho menos ciclos. Por otra parte el Greedy puede devolver conjuntos dominantes no mínimos (es decir un conjunto dominante tal que exista un conjunto con menor cantidad de nodos que también sea dominante) mientras que el Exacto absolutamente siempre va a devolver el mínimo. \\
Teniendo en cuenta estos conceptos, para saber que algoritmo nos conviene usar tendríamos que tener en cuenta varias cosas. Si sabemos que las instancias de entrada rara vez serán con mas de un nodo del mismo grado, no nos afecta significativamente que el algoritmo nos devuelva esporádicamente resultados subóptimos y necesitamos que el algoritmo sea veloz, claramente la mejor opción es el Greedy. Ya que dificilmente fallaría (porque casi siempre vendrían grafos con todos nodos de distintos grados), funciona mucho más rápido que el exacto y el caso en que falle no sería decisivo. \\
Ejemplo:\\
Un ejemplo de la vida real de este caso podría ser si quisiese elegir a los referentes políticos de cada provincia de forma tal que representen a toda la poblacion a través de la influencia en esta que ejercen . Donde influencia se entiende como gente que lo apoya y cada persona puede apoyar a mas de uno (esto se obtuvo a través de encuestas). El problema lo abstraeríamos a grafos diciendo que cada nodo es una persona y estas se relacionan si una apoya la otra (no importa quien a quien ya que entendemos que si un ciudadano apoya a un candidato político es porque este también apoyaría al ciudadano, un poco utópicos lo se). En este caso como por lo general los dirigentes políticos zonales tienen influencia gracias a los partidos políticos a los que pertencen y estos suelen diferir entre sí en la cantidad de adherentes (es muy raro que dos partidos tengan exactamente la misma cantidad de adherentes) entonces en la gran mayoría de los casos los grafos van a estar compuestos todos por nodos con distinto grado y además van a ser datos de entrada muy grandes (ya que cada provincia contiene como mínimo 100 mil habitantes) por lo tanto en este caso sería muchísimo mejor utilizar el Greedy. \\
En cambio si no me importa mucho el tiempo de proceso, se que los datos de entrada no van a ser muy grandes y necesito que el resultado sea siempre el mejor me conviene usar el exacto. Un caso de esto podría ser si mediante un mapa del circuito eléctrico de una ciudad del interior (es decir una ciudad chica) quiero obtener los puntos desde los cuales puedo llegar a todo el resto del tramado eléctrico para mejorar los transistores, para esto mediante la obtención del conjunto dominante minímo obtendríamos estos puntos y los devolveríamos. Si consideramos que mejorar cada transistor nos cuesta cientos de miles de pesos es primordial que el resultado sea exacto y como solo deseamos correrlo una vez no es tan importante el tiempo que pueda tardar. Por eso en este caso la mejor opción sería el algoritmo exacto.



\subsection{Exacto - Local Search}

\subsection{Exacto - GRASP}

\subsection{Goloso - Local Search}

\subsection{Goloso - GRASP}

\subsection{Local Search - GRASP}
