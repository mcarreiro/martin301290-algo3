\section{Casos de la Vida Real}

\subsection{Situación 1}
Supongamos que se organiza una fiesta a la que asiste mucha gente (es una fiesta democrática e inclusiva y cada uno puede invitar a todos los amigos que quiera). Como la fiesta es en un parque de diversiones jurásico en el que hay dinosaurios vivos y mutantes sueltos es necesario capacitar a la gente por cualquier accidente o imprevisto que pueda ocurrir. Ahora, resulta que la fiesta tuvo tanta difusión que asistieron alrededor de 70 mil personas. Esto desconcertó profundamente a los organizadores, ya que al ser tanta gente es muy difícil capacitarla. Por suerte el presidente del parque conocía a un grupo de investigadores de exactas (grupo 10) a los que les delegó la tarea, aclarándoles que con la única información que contaban era con una tabla de relaciones (obtenida de facebook) personas a personas, en la que una persona se relaciona con otra si y solo si son amigos.
Los investigadores abstrayeron el problema y realizaron el siguiente planteo:
-Planteemos el problema como un problema de grafos!
-Cada persona es un nodo
-Un nodo se relaciona con otro si y solo si esas personas son amigas
-Busquemos la menor cantidad de nodos para capacitar (podría pensarse como los más populares), para que luego estos nodos (o personas) capaciten a sus amigos y se logre así capacitar a toda la gente que concurrió a esta zarpada fiesta.

Buenísimo pero ahora ¿Cómo obtenemos la menor cantidad de gente a capacitar?.... Conjunto dominante!, este conjunto dominante representaría a la mínima cantidad de gente a capacitar, tal que luego esas personas al capacitar a sus amigos lograsen capacitar a toda la gente que concurrió a la fiesta.

\subsection{Situación 2}
Supongamos que en un tablero de ajedrez queremos poner algunas reinas de manera que todas las casillas estén amenazadas por al menos una de ellas y queremos usar la menor cantidad posible de damas. ¿Cuántas reinas son necesarias? y ¿cómo hay que ubicarlas? Se puede plantear este ejemplo como un problema de conjunto dominante. Para ello
consideramos el grafo donde los vértices representan las casillas de un tablero de ajedrez y dos vértices son adyacentes si una reina ubicada en la casilla u amenaza la casilla v. Un conjunto dominante mínimo D nos dice dónde hay que ubicar las reinas en el tablero para amenazar a todas las casillas. 

