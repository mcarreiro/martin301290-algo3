\section{Problema 2}

\subsection{Orden de complejidad}

Pseudocódigo:

global grafoOriginal

obtenerConjuntoDominanteMinimo(grafo g){
c = crearConj()
grafoOriginal = g
return buscar_minimo(g,c)

}

buscar_minimo(grafo, conjuntoDom){
	if(es_dominante(conjuntoDom)){
		return conjuntoDom
	}else{
		vertice = grafo.obtenerVertice(); 
		grafo = grafo.sinUno()
		return min(buscar_minimo(grafo, conjuntoDom), buscar_minimo(grafo, conjuntoDom + vertice)
	}
}

es_dominante(grafo, conjuntoDom){
	foreach(v en V(grafoOriginal)){
		if(conjuntoDom.esta?(v){
			continue
		}else{
			encontre = false
			foreach(ver en conjuntoDOm){
				if(ver.adyacentes.esta?(v))
					break
			}
			if(!encontre)
				return false
		}
	}
	return true
}

La complejidad de mi algoritmo es de 2^n. Voy a demostrarlo por inducción:

Caso Base:

n = 1, si n tiene un solo nodo ver si es dominante me cuesta O(1) ya que recorrer los vertices del grafo original es una sola iteracion y no es posible recorrer sus aristas. Luego divido en el caso en el que uso a ese nodo en el conjunto dominante y el caso en que no. El caso en que no al no tener mas nodos con cual probar me va a devolver el grafo original y el otro caso también ya que el grafo original era el que contenía únicamente a ese nodo. Ambos casos ver si el conjunto es dominante cuesta O(1) ya que tiene a lo sumo una iteración para hacer. Por lo tanto el algoritmo costaría O(1) que es igual a O(2¹*1³).

Hipótesis inductiva:

Supongo que con n nodos la complejidad es 2^n * n³

Paso inductivo:

Quiero ver que con n+1 nodos la complejidad pertenece a 2^(n+1)*(n+1)³

Ver si el conjunto vacío es dominante me cuesta O(1) ya que en la primera iteración del grafoOriginal no es posible encontrar ningún vertice en el conjunto dominante o adyacente a él. Luego obtengo un nodo del grafo (grafo en el que me guarde todos los vertices, no el origianl) y busco el minimo conjunto dominante agregando ese nodo al conjunto o no, esto por hipótesis inductiva me cuesta 2 * (2^n * n³), ya que tengo que calcular 2 veces el conjunto minimo para n nodos. Por lo tanto la complejidad me termina costando  2^(n+1)*n³ y esto pertence a 2^(n+1)*(n+1)³. Con lo cual queda demostrada la complejidad.

Notar que esta complejidad esta por encima de la complejidad exacta, ya que el n³ variaría en cada iteración dependiendo de la cantidad de nodos en el conjuntoDominante.
