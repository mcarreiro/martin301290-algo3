\documentclass[8pt, A4]{article}

%Margenes de la pagina.  otra opcion, usar \usepackage{a4wide}
\usepackage[paper=a4paper, left=0.8cm, right=0.8cm, bottom=1.3cm, top=0.9cm]{geometry}
\usepackage{color}

%este paquete permite incluir acentos.  Notar que espera un formato ANSI-blah de archivo.  Si en lugar de eso se tiene un utf8 (usual en los linux), entonces usar \usepackage[utf8]{inputenc}
\usepackage[utf8]{inputenc}

%Este paquete es para que algunos titulos (como Tabla de Contenidos) esten en castellano
\usepackage[spanish]{babel}

%El siguiente paquete permite escribir la caratula facilmente
\usepackage{caratula}

\usepackage{aed2-symb,aed2-itef,aed2-tad,aed2-tokenizer,modulos_diseno, ./algorithms/clrscode3e}
\usepackage{framed}
\usepackage{amsmath}

\usepackage{graphicx}

%Datos para la caratula
\materia{Algoritmos y Estructuras de Datos III}

\titulo{Trabajo Pr\'actico 2}


\integrante{Ortiz de Zarate, Juan Manuel}{403/10}{jmanuoz@gmail.com}
\integrante{Martelletti, Pablo}{849/11}{pmartelletti@gmail.com}
\integrante{Kujawski, Kevin}{459/10}{kevinkuja@gmail.com}
\integrante{Carreiro, Martin}{45/10}{carreiromartin@gmail.com}

\begin{document}
%numero de grupo
{\hfill\Huge 10} %\includegraphics[width=2cm]{./patologico.png} \\ 
%esto construye la caractula
\maketitle 

 
 \tableofcontents

 \newpage

%\section{Introducci\'on}
En el siguiente trabajo presentaremos la resolución a tres problemas que se nos dio a resolver. Mostraremos para cada uno
el algoritmo que utilizamos para resolverlo describiendo que técnicas de las vistas en clase utilizamos, mostraremos también
el análisis de complejidad de cada algoritmo y mostraremos gráficamente los resultados de las mediciones para ver si se cumple el análisis 
teórico.


%  \newpage
\section{Problema 1}

\subsection{Enunciado}
El enunciado nos plantea una situación en donde tenemos un edificio con n pisos y personas en cada piso que quiere ir a planta baja. Para poder hacerlo, el edificio provee un 
ascensor que deberá buscar a las personas para poder bajarlas. 
Este ascensor posee energía y capacidad limitada que no le permite recorrer siempre todos los pisos y levantar todas las personas, por lo tanto queremos maximizar la cantidad
de personas a descender del edificio dado la cantidad de personas por piso, su energía y la capacidad.

\subsection{Soluci\'on}
La solución planteada utiliza Programación Dinámica a través de decisiones. Es decir, planteamos el problema de forma tal que en vez de maximizar la cantidad de personas que 
se pueden descender, encontramos el máximo valor posible que va a estar ubicado entre cero y la cantidad total de personas en el edificio.\\
A continuación se explicará cómo se resolvió el problema:\\
Primero obtenemos la cantidad total de personas recorriendo el edificio dado, y buscaremos ese dicho máximo valor.
Para ello, utilizaremos la conocida búsqueda binaria en la cantidad de personas en el edificio, hasta encontrar el valor tal que el siguiente no sea posible levantarlo y sin embargo el número que estoy evaluando si.\\
Por lo tanto lo único que queda es ver si se puede levantar ese valor obtenido por parámetro


Para solucionar el problema pensamos en crear una función que reciba todos los datos del enunciado (capacidad, energia, pisos y cantidad de personas en cada uno) mas un entero que represente si es posible levantar esa cantidad de personas. Osea la funcion devolvería un boleean. Entonces lo que hacemos es averiguar cuanta gente hay en total en el edificio y mediante un busqueda binaria (entre 0 y el total) le vamos preguntando a la función si es posible levantar esa cantidad de gente hasta encontrar el punto de quiebre (cuando es true y el siguiente valor ya es false) o el total y finalmente devolvemos el valor que encontramos.

\subsection{Pseudoc\'odigo}
En código del ejercicio básicamente hace una cosa: lee el archivo de entrada, y procesa las lineas y va creando las distintas instancias del ejercicio. La lógica del problema está toda delegada en la clase Grafo, que es la que se encarga de modelar las localidades y sus enlaces (el grafo en sí), y luego, una vez modelado, de crear el propio AGM. El agm está implementado a partir del algoritmo de prim, que busca para cada vertice nuevo agregado, la arista de menor peso que lo conecte con un vertice que no esté en el grafo y la agrega.
La lógica del ejercicio podría dividirse, de acuerdo a lo recién mencionado, de la siguiente manera:

\begin{verbatim}
Grafo CrearGrafoAPartirInstancia():
	Grafo g;
	para cada enlace de la instancia:
		agrego el vertice 1
		agrego el vertice 2
		agrego la arista entre v1 y v2
	devuelvo el grafo 
\end{verbatim}

\begin{verbatim}
Grafo getAgm():
	Grafo agm;
	agrego un vertice al azar del grafo al agm
	marco el vertice agregado como visitado
	agrego las aristas del vertice visitado a la cola de prioridad
	para cada vertice en grafo:
		obtengoLaAristaDeMenorPesoDeLaCola()
		miro los dos vertices de la arista
		si alguno no fue visitado:
			agrego la arista minima al agm
			marco el vertice como visitado
			agrego las aristas del vertice a la cola de prioridad
	devuelvo el agm
\end{verbatim}

\subsection{Analisís de complejidad}	
Para averiguar cuanta gente hay en todo el edificio tenemos que recorrer todo el vector 'pisos' e ir sumando la cantidad de gente en cada piso. Eso nos cuesto O(n) donde n es la cantidad de pisos. Luego hacemos una búsqueda binaria sobre la cantidad total de gente que mediante la función sePuedeLevantar busca el máximo de gente que es posible cargar. La complejidad de la búsqueda binaria es de O(log cantGente) (esto es sabido, porque ya lo vimos en algoritmos y estructura de datos 2) y la complejidad de la función sePuedeLevantar en el peor caso (sería cuando tiene que irse hasta el último piso para poder levantar la cantidad de gente solicitada y tiene energía suficiente para levantarlos a todos) es de: \\
	O( $\sum\limits_{i=0}^{n} { ( \lceil (pisos[i]/capacidad) \rceil  + i}$ ) ) \\
Esto es porque por cada piso al que voy tengo que ir tantas veces tal que levante el total de gente de ese piso (eso es $\lceil$pisos[i]/capacidad$\rceil$) y luego en caso que me haya sobrado espacio voy recorriendo todos los pisos inferiores para llenar la capacidad restante.
Finalmente la complejidad total del ejercicio nos queda:\\
O( n + Log(g) * $\sum\limits_{i=0}^{n} { ( \lceil (pisos[i]/capacidad) \rceil  + i ) }$ ) \\
Donde n es la cantidad total de pisos y g la cantidad total de gente en todo el edificio.
\subsection{Tests y Gráficos}
Con respecto a los tests realizados, los mismos no se hicieron en cuanto a la complejidad (ya que al utilizar estructuras primitivas de java, podemos asegurar que la complejidad de cada una de sus operaciones es la que aparece en la documentación de ellas y por tanto, es la que detallamos en el apartado anterior), sino en cuanto a la cantidad de ciclos que realiza cada solución (más precisamente, la creación del árbol generador mínimo) para distintas instancias, de acuerdo a la cantidad de vértices (localidades) y aristas (enlaces) que posee cada una.
Como podemos apreciar en los siguientes gráficos, la cantidad de ciclos, para grafos en donde la cantidad de aristas es la misma, es directamente proporcional a la cantidad de vértices que contenga el mismo (figura 1). Es cierto que en algunos casos ésto no se cumple pero, en el caso promedio, a más cantidad de vértices, con igual cantidad de aristas, más ciclos tendrá que hacer nuestro algoritmo para encontrar el AGM válido. Los casos extremos serían aquellos en donde las aristas se insertan ordenados de la misma forma que los leerá nuestro algoritmo, donde la cantidad de ciclos se corresponde con la cantidad de vértices. Por el contrario, el peor caso se da cuando hay muchas aristas de 2 vértices que ya fueron visitados, de menor peso de aquellas que contengan 1 vértice visitado y otro sin visitar. De ésta forma, el algoritmo hará tantos ciclos como aristas tenga el vértice para, recién en la último, procesar el vértice y agregarlo al AGM.
\begin {center}
\includegraphics[width=8cm]{./graficos/grafico_vfijo.png}
% grafico.eps: 0x0 pixel, 300dpi, 0.00x0.00 cm, bb=50 50 410 302
\end {center} 
Por otro lado, y con respecto a la figura 2, donde la cantidad de aristas está fijo, vemos que la cantidad de ciclos que se llevan a cabo es netamente aleatorio, ya que, de acuerdo a cómo estén distribuidas las distintas aristas y sus pesos, es posible crear el AGM en n pasos, siendo n la cantida de vértices, o bien, en n x e(n), siendo e(n) la cantidad de aristas de n. El primer caso se daría sólo cuando el grafo que analizamos posee aristas minimas sin descubrir en todos los pasos, mientras que el peor caso, que sería el de recorrer todas las aristas del vértice, es el mismo que analizamos en el párrafo anterior.
\begin {center}
\includegraphics[width=8cm]{./graficos/grafico_efijo.png}
% grafico.eps: 0x0 pixel, 300dpi, 0.00x0.00 cm, bb=50 50 410 302
\end {center} 


\subsection{Conclusiones}
A partir del problema dado, hemos podido modelar a partir de la teoría de grafos. De ésta forma, e interpretando qué es lo que nos pide el enunciado, hemos podido resolver el problema a partir de un algoritmo que es conocido y, de esa forma, podemos afirmar que estamos dando la solución correcta. 
En ésta caso, modelamos nuestras localidades y los distintos precios que costaban unirlas a partir de un grafo conexo, no dirigido. Para éstos grafos, es posible hallar un grafo de peso mínimo que conecte todas las ciudades, conocido también como árbol generador mínimo (agm). Precisamente, ésto es lo que nos pedía el enunciado: hallar la forma de conectar todas las ciudades de forma tal que el coste de unirlas todas sea mínimo. Con nuestro planteo, logramos responder dicho problema, teniendo en cuenta, además, la complejidad pedida para resolverlo. La misma fue lograda no sólo implementando un algoritmo conocido, como es el de prim, sino que se fue cuidadoso a la hora de elegir las estructuras de datos utilizadas para mantenernos dentro de la cota estipulada.

  \newpage
 \section{Algoritmo Exacto}

\subsection{Enunciado}
En este ejercicio se nos solicita buscar el conjunto dominante mínimo y óptimo dado un grafo cualquiera. Este es un problema del tipo NP completo 
para el cual aún no se encontró forma de resolverlo polinomialmente pero tampoco se demostró que no sea posible solucionarlo con dicha complejidad.

\subsection{Soluci\'on}
Como todavía no se halló algoritmo alguno para resolverlo polinomialmente y nosotros no somos investigadores/iluminados (aún) decidimos resolverlo
de manera exponencial. Para esto utilizamos el popular método de backtracking (dicho en criollo) de quedarme con la mejor opción entre poner o no un nodo en el 
conjunto dominante. Este procedimiento lo que hace, básicamente, es analizar todas las soluciones posibles y agarrar la mejor de ellas.  \\
No nos pareció significativo agregarle memorization ya que su complejidad no mejoriría de forma considerable debido a que la complejidad del algoritmo reside en calcular cada solución posible y no en el recalculo de las mismas, además de que ocuparía mas memoria. Tampoco nos pareció imperante preocuparnos por la complejidad de la función que chequea si el conjunto recibido es dominante, ya que, mientras sea polinimal, va a ser despreciable al lado de la complejidad de analizar todas las soluciones (que como dijimos es exponencial). \\
Finalmente queremos resaltar que cualquier optimización conocida para este algoritmo no haría mas que mejorar la complejidad para ciertos tipos de casos, como el ejercicio no especifica que los grafos a recibir cumplan ciertos parametros o restricciones, todo tipo de perfeccionamiento que le implementemos va a dar lo mismo para las consignas del ejercicio.

\subsection{Pseudocódigo}

global grafoOriginal

\begin{codebox}
\Procname{$\proc{obtenerConjuntoDominanteMinimo}$ (\textbf{in} $Grafo$)}{conjuntoDom}{Conj}
\li	c = crearConj()
\li	grafoOriginal = Grafo
\li	\textbf{return} buscarMinimo(Grafo,c)
\end{codebox}

\begin{codebox}
\Procname{$\proc{buscarMinimo}$(\textbf{in} $Grafo$, \textbf{in} $conjuntoDom$)}{conjuntoDom}{Conj}
\li\textbf{Si} esDominante(conjuntoDom) \textbf{Hacer:} \Do
\li		\textbf{return} conjuntoDom 
\End
\li	\textbf{Si no}  \Do
\li		vertice = grafo.obtenerVertice(); 
\li		grafo = grafo.sinUno(vertice);
\li		conjunto1 = buscarMinimo(grafo, conjuntoDom);
\li		if($|$conjunto1$|$ == 1) \textbf{return} conjunto1
\li		conjunto2 = buscarMinimo(grafo, conjuntoDom + vertice);
\li		if($|$conjunto2$|$ == 1) \textbf{return} conjunto2
\li		\textbf{return} min(conjunto1, conjunto2)	
\End

\end{codebox}

\begin{codebox}
\Procname{$\proc{esDominante}$(\textbf{in} $Grafo$, \textbf{in} $conjuntoDom$)}{esDominante}{Boolean}
\li \textbf{Para} cada v en V(grafoOriginal) \Do
\li \textbf{Si} conjuntoDom.esta?(v) \textbf{Hacer:} \Do
\li			continue 
		\End
\li \textbf{Si no}  \Do
\li			encontre = false
\li \textbf{Para} cada ver en conjuntoDOm \Do
\li	\textbf{Si} ver.adyacentes.esta?(v) \textbf{Hacer:} \Do	
\li			encontre = true
\li			break
			\End
	\End		
\li	\textbf{Si} !encontre \textbf{Hacer:} \Do				
\li		\textbf{return} false
			\End		
	\End
\li	\textbf{return} true
\End
\end{codebox}

\subsection{Análisis de Complejidad}

La complejidad de mi algoritmo es de $2^n$ * $n^3$. Ya que hace $2^n$ recursiones y en cada una de ellas ver si el conjunto formado es dominante cuesta a lo sumo $n^3$. Voy a demostrar por inducción la parte exponencial:\\

\underline{Caso Base:}\\

n = 1, si n tiene un solo nodo ver si es dominante me cuesta O(1) ya que recorrer los vertices del grafo original es una sola iteracion y no es posible recorrer sus aristas. Luego divido en el caso en el que uso a ese nodo en el conjunto dominante y el caso en que no. El caso en que no al no tener mas nodos con cual probar me va a devolver el grafo original y el otro caso también ya que el grafo original era el que contenía únicamente a ese nodo. Ambos casos ver si el conjunto es dominante cuesta O(1) ya que tiene a lo sumo una iteración para hacer. Por lo tanto el algoritmo costaría O(1) que es igual a O($2^1*1^3$).\\

\underline{Hipótesis inductiva:}\\

Supongo que con n nodos la cantidad de recursiones es $2^n$ \\

\underline{Paso inductivo:}\\

Quiero ver que con n+1 nodos la cantidad de recursiones será $2^{n+1}$\\

Ver si el conjunto vacío es dominante me cuesta O(1) ya que en la primera iteración del grafoOriginal no es posible encontrar ningún vertice en el conjunto dominante o adyacente a él. Luego obtengo un nodo del grafo (grafo en el que me guarde todos los vertices, no el origianl) y busco el minimo conjunto dominante agregando ese nodo al conjunto o no, esto por hipótesis inductiva me cuesta 2 * ($2^n$), ya que tengo que calcular 2 veces el conjunto dominante mínimo para n nodos. Por lo tanto la complejidad me termina costando  $2^{n+1}$. Con lo cual queda demostrada la complejidad.\\
\\
Ahora voy a demostrar que verificar si el conjunto recibido es dominante cuesta  O($n^3$):\\
\\
La iteración principal la hace sobre los nodos del grafo sin modificar el grafo sobre el que se itera y sin volver hacia atrás en la iteración, por lo tanto realizará a lo sumo n pasos donde n = $|$V(grafo)$|$.\\
Dentro de cada ciclo realiza todas operaciones básicas que cuestan O(1), utiliza la función de ArrayList 'contains' que toma O($|$conjunto recibido$|$)$^{1}$  e itera sobre el conjunto recibido (el conjunto que queremos ver si es dominante) para ver si el nodo del grafo original que tome pertence a algun adyacente o es parte del conjunto. En el caso que el grafo del que deseamos obtener el conjunto dominante mínimo contenga un clique de n/2 nodos y cada uno de los nodos que estan fuera del clique se relacionen con todos los nodos del clique menos uno, como el caso que podemos ver en la siguiente imagen:\\

 \begin {center}
\includegraphics[width=12cm]{./graficos/grafo-clique-N-sobre-2.png}
% grafico.eps: 0x0 pixel, 300dpi, 0.00x0.00 cm, bb=50 50 410 302
\end {center} 

Ver si el nodo que tomé pertenece a ese conjunto o es adyacente puede tomarme (n/2) ciclos (en el caso que el conjunto recibido sea el clique), donde además ver si es adyacente a cada uno de los nodos me tomaría ((n/2) - 1) comparaciones. Por lo tanto la complejidad total sería (n/2) * ((n/2) -1) que pertenece a O($n^2$) y esto podría llegar a hacerlo un total de (n/2) veces en la iteración principal, por lo tanto la complejidad total me queda O($n^3$). Notar que no es posbile que la complejidad de este for anidado supere esa complejidad ya que el conjunto dominante no puede tener mas de n nodos y cada nodo de este conjunto no puede tener mas de n-1 nodos adyacentes.


\subsection{Peor Caso}

Como el algoritmo es exacto y recorre todas las soluciones posibles siempre obtiene la mejor de ellas. Por lo tanto no existe una 'peor' instancia
en la que la solución devuelta sea sub-óptima, siempre devuelve la mejor. 

\subsection{Tests y análisis}
En los siguientes gráfico podemos observar que la cantidad de ciclos y el tiempo crecen notablemente a medida que el grafo tiene cada vez mas nodos.\\
Esto es debido a que, como dijimos anteriormente, la complejidad es exponencial en el tamaño de la entrada (en este caso, la cantidad de nodos 
del grafo) y no existe un mejor o peor en caso en el cual la complejidad sea mucho menor a O($2^n*n^3$). Por esto es que tanto en el grafico de tiempos
 como en el de ciclos se puede observar claramente que la complejidad es exponencial.

\begin {center}
\includegraphics[width=12cm]{./graficos/exacto.png}
% grafico.eps: 0x0 pixel, 300dpi, 0.00x0.00 cm, bb=50 50 410 302
\end {center} 

\begin {center}
\includegraphics[width=12cm]{./graficos/exactoConTiempo.png}
% grafico.eps: 0x0 pixel, 300dpi, 0.00x0.00 cm, bb=50 50 410 302
\end {center}

  \newpage
\section{Algoritmo Goloso}

\subsection{Soluci\'on}

Proponemos una solución heurística golosa para encontrar un conjunto dominante de un grafo lo mas pequeño posible con una complejidad polinomial.\\
Para ello nos basamos en una estrategia de selección de nodos dominantes, la cual consiste en elegir el nodo que mas adyacentes no cubiertas tenga (es decir, que todavía no fueron dominadas) y verificando en cada iteración si el conjunto actual es dominante. En base a diferentes estrategias y tipos de grafos (estrella, bipartitos, completos, estrellas unidas, caminos, ciclos) que fuimos observando, elegimos esta ya que fue la que más nos convenció y mejores resultados nos dió debido a que siempre intenta seleccionar el nodo que mas pueda dominar a otros nodos reduciendo el conjunto de los no cubiertos y acercándose a una mejor solución del problema. 

\subsection{Pseudocódigo}

\begin{codebox}
\Procname{$\proc{MCDGreedy}$ (\textbf{in} $Grafo$)}{conjuntoDomGoloso}{ConjDeVértices}
\li	vértices = lista de vértices del grafo	\RComment O(n)
\li	dominantes = conjunto vacio de vértices	
\li	\textbf{Mientras} no estén TodasCubiertas(vértices) \Do \RComment O($n$)
\li 		Ordeno los vértices por la cantidad adyacentes que tenga no cubiertas (sin dominar), de mayor a menor. \RComment O(n*log(n))
\li 		Elijo como dominante al primero de la lista y lo agrego al conjunto de \textbf{dominantes} \RComment O(n)
\li 		Saco de la lista de vértices al elegido \RComment O(n)
\li 		ActualizarGradoSinDominar(elegido) \RComment Actualizo los nodos del grafo,\\ disminuyendo la cantidad de \textit{grado sin dominar} de los adyacentes al elegido, y de los adyacentes a estos.  O($n^2$)
\End
\li	\textbf{return} dominantes
\end{codebox}

\begin{codebox}
\Procname{$\proc{TodasCubiertas}$ (\textbf{in} $ListaDeVertices$)}{result}{Boolean}
\li 	\textbf{Para todos} los vértices en la lista \Do
\li 		\textbf{Si} el vértice no esta dominado \Do
\li 			\textbf{return} falso \End \End
\li 	\textbf{return} verdadero
\end{codebox}

\begin{codebox}
\Procname{$\proc{ActualizarGradoSinDominar}$ (\textbf{in} $Vertice$ $elegido$)}{}{}
\li 	Marcar al vértice elegido como dominado.
\li 	unionDeAdyacentes = lista de vértices vacía, que luego se usará para actualizar.
\li 	\textbf{Para todos} los vértices en la lista de adyacentes al \textbf{elegido} \Do
\li 		\textbf{Si} el vértice \textit{elegido} no estaba dominado \Do
\li 			Decremento en uno el grado de adyacentes sin dominar del nodo actual \End
\li 		\textbf{Si} el nodo actual \textit{elegido} no estaba dominado \Do
\li 			Agrego los nodos adyacentes a la lista \textit{unionDeAdyacentes} \End
\li 		Marco al nodo actual como dominado. \End
\li 	\textbf{Para todos} los vértices en la lista unionDeAdyacentes \Do 
\li 			Decremento en uno el grado de adyacentes sin dominar del nodo actual \End %\RComment Recorro unionDeAdyacentes para actualizar\\  el grado de nodos adyacentes sin dominar
\end{codebox}

\subsection{Análisis de Complejidad}

La complejidad del algoritmo es de O($n^3$)

El algoritmo comienza creando una lista de vértices en O(n).\\
Luego entra en un ciclo que como máximo será lineal en la cantidad de vértices porque en cada ciclo saca un vértice y la cantidad de vértices es finita. En cada iteración deberá comprobar si el conjunto actual de vértices elegidos domina todo el grafo en O($n$), ordenar todos los vértices en O(n*log(n)), elegir un vértice en O(1), removerlo de la lista en O(n) y finalmente actualizar el atributo de los grados sin dominar adyacentes de cada nodo en O($n^2$), esto es porque para los adyacentes del nodo elegido y a su vez para los adyacentes de estos tengo que bajar en uno este atributo, recorrer los adyacentes me lleva O(n) y por cada adyacente recorrer sus adyacentes también me lleva O(n), O(n) * O(n) = O($n^2$).\\

Por lo tanto, la complejidad nos queda:\\
O(n) + O(n) * (O($n^2$) + O(n*log(n) + O(n) + O($n^2$)) = O($n^3$)\\\\

La complejidad final del algoritmo goloso es de O($n^3$), cumpliendo el objetivo de ser polinomial.

\subsection{Mejor Caso}

Las familias de grafos en donde la solución es la mejor son en los grafos bipartitos completos, triangulos unidos, estrellas, arboles binarios, arbol de cliques. Por que? Porque en cada ordenamiento y selección de nodos, el nodo con más grado sin dominar es exactemente el que hay que elegir. EXPLICAR MAS

\subsection{Peor Caso}

Al ser una heurística, en cierto casos la solución no es la óptima que podría dar un algoritmo exacto, pero aún asi es correcta.
Los peores casos, donde se produce una diferencia en el tamaño de conjunto dominante respecto a la optima, se suelen dar en grafos en los que varios nodos tienen el mismo grado o hay pequeña diferencia, y esto es debido que para la elección del vertice dominante en cada iteración nos quedamos con el que mayor grado de nodos adyacentes no cubiertos tenga y si hay varios con esta misma característica puede pasar que el nodo seleccionado no sea conveniente a futuro para llegar a una solución óptima.

En los siguientes ejemplos de grafos se puede apreciar mejor:

\begin {center}
\includegraphics[width=8cm]{./graficos/grafo.png}
% grafico.eps: 0x0 pixel, 300dpi, 0.00x0.00 cm, bb=50 50 410 302
\end {center} 
La solución optima proporcionada por el algoritmo exacto es {6,2}, mientras que el goloso devuelve {4,3,5}, ya que como los nodos 6,4 y 5 tienen el mismo grado, al momento de la elección se decide por el 4 provocando esta diferencia en el tamaño del conjunto con respecto a la solución óptima.

\begin {center}
\includegraphics[width=8cm]{./graficos/grafo_camino.png}
% grafico.eps: 0x0 pixel, 300dpi, 0.00x0.00 cm, bb=50 50 410 302
\end {center} 
La solución óptima proporcionada por el algoritmo exacto es {2,4}, mientras que el goloso podría devolver {3,2,5}, ya que como los nodos 2, 3 y 4 tienen el mismo grado, si se decide por el 3, dejaría a todos los demás nodos con 1 grado sin dominar y faltando los dos extremos, provocando que si o si se necesite cubrirlos o elegirlos como dominantes, haciendo que el conjunto tenga tamaño 3 y no 2 como el exacto.

\subsubsection {Familias}
Las familias de grafos que encontramos que el algoritmo goloso da un resultado no optimo frente al exacto fue con los Möbius–Kantor y Grid. Esto suele pasar por que son regulares (3-regular) y casi regulares (4-regular internamente, 3-regular en los bordes) respectivamente y al momento de la elección del mejor vértice, la estrategia solo distingue por grado de adyacentes sin dominar por lo tanto seleccionará por el orden de la etiqueta de los nodos, las cuales son arbitrarias. 

Considerando que Möbius–Kantor se puede reducir a un circuito, nos pareció mas interesante enfocarnos en los grafos Grid, ya que es más amplia la diferencia de resultados entre los dos algoritmos, además esta familia de grafos son de más utilidad, ya que equivalen tableros representando cada celda con un nodo.

En el siguiente ejemplo, para un grafo Grid de 5x5 con 25 nodos, se puede notar la diferencia entre cantidad de elementos de los conjunto dominante resultantes, el goloso devuelve 9 nodos {6, 18, 21, 9, 1, 11, 20, 4, 23}, mientras que el exacto 7 nodos {19, 18, 10, 7, 4, 1, 21}, y esto se debe principalmente a que el goloso eligirá primero los nodos internos de grado 4 hasta que tengan el mismo grado sin dominar que los del borde y luego seleccionará los que faltan para dominar a todos, mientras que la solución exacta dependia más de los nodos en los bordes que los interiores.
 
\includegraphics[width=8cm]{./graficos/grid_5x5_goloso.png}
% grafico.eps: 0x0 pixel, 300dpi, 0.00x0.00 cm, bb=50 50 410 302
\includegraphics[width=8cm]{./graficos/grid_5x5_exacto.png}
% grafico.eps: 0x0 pixel, 300dpi, 0.00x0.00 cm, bb=50 50 410 302


\begin {center}
\includegraphics[width=19cm]{./graficos/goloso-900nodos.png}
% grafico.eps: 0x0 pixel, 300dpi, 0.00x0.00 cm, bb=50 50 410 302
\end {center} 
\begin {center}
 \begin{tabular}{ l | l l l l l l l l l l l l l l l l l l l l l l l l l l l l l}
m/n & 2 & 3 & 4 & 5 & 6 & 7 & 8 & 9 & 10 & 11 & 12 & 13 & 14 & 15 & 16 & 17 & 18 & 19 & 20 & 21 & 22 & 23 & 24 & 25 & 26 & 27 & 28 & 29 \\ \hline
1 & 1 & 1 & 1 & 2 & 2 & 2 & 3 & 4 & 3 & 4 & 5 & 6 & 6 & 6 & 7 & 7 & 7 & 6 & 7 & 8 & 7 & 8 & 9 & 9 & 9 & 10 & 10 & 10 & 11\\
2 & 1 & 2 & 2 & 3 & 4 & 4 & 5 & 6 & 7 & 8 & 8 & 9 & 10 & 11 & 11 & 12 & 14 & 13 & 13 & 14 & 16 & 16 & 16 & 18 & 18 & 18 & 20 & 20 & 20 \\
3 & 1 & 2 & 3 & 4 & 5 & 6 & 8 & 8 & 9 & 10 & 11 & 12 & 14 & 14 & 15 & 17 & 17 & 18 & 19 & 20 & 20 & 22 & 23 & 24 & 26 & 26 & 27 & 29 & 29  \\
4 & 2 & 3 & 4 & 6 & 6 & 8 & 10 & 10 & 11 & 13 & 15 & 15 & 18 & 18 & 20 & 20 & 21 & 25 & 26 & 26 & 27 & 28 & 31 & 31 & 34 & 34 & 35 & 38 & 38 \\
5 & 2 & 4 & 5 & 6 & 9 & 9 & 12 & 13 & 14 & 15 & 19 & 16 & 21 & 21 & 25 & 27 & 26 & 27 & 30 & 31 & 31 & 33 & 38 & 35 & 41 & 39 & 43 & 45 & 47  \\
6 & 2 & 4 & 6 & 8 & 10 & 10 & 14 & 16 & 18 & 18 & 21 & 26 & 23 & 27 & 27 & 28 & 31 & 33 & 35 & 37 & 39 & 41 & 45 & 46 & 47 & 49 & 50 & 51 & 53  \\
7 & 3 & 5 & 6 & 9 & 13 & 13 & 15 & 19 & 20 & 21 & 25 & 24 & 29 & 31 & 31 & 36 & 36 & 39 & 43 & 45 & 47 & 46 & 48 & 48 & 52 & 55 & 56 & 58 & 61  \\
8 & 4 & 5 & 8 & 12 & 12 & 16 & 15 & 21 & 21 & 25 & 29 & 28 & 31 & 35 & 38 & 42 & 43 & 43 & 47 & 47 & 52 & 51 & 55 & 57 & 63 & 68 & 70 & 74 & 71  \\
9 & 3 & 6 & 9 & 11 & 13 & 16 & 21 & 23 & 28 & 28 & 31 & 32 & 36 & 42 & 44 & 47 & 51 & 49 & 53 & 54 & 58 & 59 & 66 & 64 & 71 & 76 & 75 & 82 & 77  \\
10 & 4 & 7 & 10 & 14 & 15 & 19 & 20 & 27 & 28 & 30 & 35 & 34 & 39 & 45 & 44 & 52 & 53 & 55 & 58 & 60 & 67 & 65 & 71 & 74 & 76 & 84 & 84 & 89 & 91  \\
11 & 5 & 8 & 11 & 15 & 17 & 21 & 23 & 29 & 31 & 33 & 39 & 39 & 42 & 49 & 46 & 55 & 60 & 59 & 66 & 68 & 73 & 70 & 77 & 85 & 93 & 91 & 89 & 101 & 100  \\
12 & 6 & 8 & 11 & 16 & 17 & 21 & 26 & 31 & 34 & 36 & 43 & 45 & 48 & 51 & 53 & 60 & 67 & 66 & 71 & 69 & 78 & 82 & 84 & 93 & 94 & 91 & 104 & 109 & 109  \\
13 & 6 & 8 & 14 & 18 & 18 & 22 & 25 & 34 & 39 & 43 & 45 & 49 & 53 & 54 & 58 & 66 & 72 & 71 & 77 & 76 & 87 & 87 & 88 & 98 & 104 & 101 & 108 & 114 & 115  \\
14 & 6 & 9 & 14 & 16 & 21 & 23 & 28 & 34 & 39 & 43 & 49 & 51 & 54 & 57 & 63 & 71 & 78 & 75 & 84 & 86 & 85 & 87 & 95 & 105 & 109 & 108 & 120 & 124 & 124  \\
15 & 7 & 9 & 15 & 18 & 21 & 28 & 31 & 40 & 41 & 47 & 51 & 53 & 61 & 64 & 65 & 75 & 82 & 82 & 90 & 88 & 94 & 98 & 103 & 109 & 122 & 118 & 127 & 134 & 130  \\
16 & 7 & 10 & 16 & 19 & 22 & 31 & 32 & 40 & 45 & 51 & 54 & 59 & 63 & 67 & 69 & 80 & 85 & 81 & 90 & 94 & 101 & 107 & 109 & 119 & 131 & 124 & 135 & 142 & 138  \\
17 & 7 & 11 & 17 & 24 & 24 & 29 & 36 & 41 & 48 & 56 & 56 & 61 & 65 & 69 & 77 & 84 & 83 & 89 & 97 & 103 & 103 & 114 & 113 & 127 & 136 & 127 & 142 & 150 & 147  \\
18 & 6 & 11 & 18 & 24 & 25 & 33 & 37 & 40 & 50 & 58 & 61 & 64 & 69 & 76 & 81 & 87 & 89 & 95 & 98 & 108 & 110 & 118 & 120 & 133 & 138 & 137 & 148 & 160 & 156 \\
19 & 7 & 12 & 19 & 27 & 25 & 35 & 39 & 45 & 52 & 61 & 64 & 69 & 74 & 79 & 85 & 94 & 94 & 99 & 109 & 113 & 116 & 126 & 130 & 141 & 146 & 144 & 157 & 168 & 165  \\
20 & 8 & 12 & 19 & 28 & 29 & 35 & 41 & 46 & 53 & 62 & 66 & 73 & 77 & 84 & 90 & 97 & 94 & 106 & 112 & 119 & 124 & 134 & 137 & 147 & 153 & 152 & 165 & 175 & 175  \\
21 & 7 & 13 & 22 & 30 & 29 & 37 & 46 & 49 & 58 & 67 & 69 & 75 & 81 & 84 & 91 & 101 & 100 & 109 & 121 & 124 & 131 & 139 & 143 & 154 & 162 & 161 & 173 & 180 & 188  \\
22 & 8 & 13 & 22 & 31 & 31 & 40 & 49 & 53 & 61 & 69 & 74 & 77 & 83 & 88 & 95 & 104 & 105 & 112 & 127 & 131 & 137 & 142 & 152 & 166 & 167 & 171 & 179 & 187 & 193  \\
23 & 9 & 13 & 23 & 33 & 31 & 38 & 51 & 56 & 64 & 71 & 76 & 83 & 90 & 93 & 97 & 109 & 110 & 119 & 133 & 139 & 139 & 153 & 158 & 170 & 181 & 178 & 188 & 192 & 203  \\
24 & 9 & 13 & 24 & 31 & 33 & 42 & 51 & 58 & 64 & 77 & 79 & 87 & 92 & 92 & 105 & 114 & 115 & 133 & 136 & 144 & 151 & 162 & 165 & 176 & 172 & 189 & 195 & 199 & 211  \\
25 & 9 & 14 & 25 & 33 & 36 & 42 & 53 & 57 & 66 & 76 & 80 & 87 & 96 & 100 & 108 & 118 & 119 & 133 & 142 & 149 & 158 & 163 & 169 & 180 & 182 & 195 & 200 & 217 & 210  \\
26 & 10 & 16 & 26 & 38 & 39 & 46 & 55 & 58 & 72 & 83 & 85 & 91 & 102 & 105 & 110 & 122 & 125 & 140 & 151 & 152 & 164 & 172 & 183 & 189 & 189 & 203 & 203 & 217 & 218  \\
27 & 10 & 16 & 27 & 38 & 43 & 45 & 56 & 60 & 71 & 83 & 89 & 97 & 107 & 110 & 114 & 126 & 129 & 141 & 156 & 157 & 166 & 177 & 183 & 197 & 200 & 217 & 217 & 223 & 226  \\
28 & 10 & 17 & 28 & 37 & 44 & 47 & 61 & 65 & 75 & 87 & 88 & 102 & 109 & 116 & 117 & 129 & 135 & 151 & 162 & 165 & 170 & 182 & 190 & 200 & 204 & 216 & 222 & 232 & 236  \\
29 & 11 & 18 & 29 & 40 & 40 & 51 & 59 & 68 & 78 & 78 & 94 & 106 & 114 & 120 & 121 & 139 & 139 & 156 & 170 & 183 & 179 & 189 & 192 & 213 & 211 & 219 & 232 & 245 & 248  \\
\end {tabular} 
\end {center} 

---
Mejores casos: ESTRELLA
Peores casos: GRID

  \newpage
%\section{Apéndice}

$ ^{1}$http://docs.oracle.com/javase/6/docs/api/java/util/ArrayList.html\\
$ ^{2}$http://www1.combinatorics.org/Volume\_18/PDF/v18i1p141.pdf\\
$ ^{3}$http://cms.dm.uba.ar/academico/carreras/licenciatura/tesis/2012/Moyano\_Veronica.pdf\\

%  \newpage

\end{document}
